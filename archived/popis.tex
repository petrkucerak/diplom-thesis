
\catcode`<=13
\def<#1>{\hbox{$\langle$\it#1\/$\rangle$}}
\toksapp\everytt{\catcode`<=13} \toksapp\everyintt{\catcode`<=13}

\chap Části dokumentu

Tento dokument nemůže sloužit jako návod k~použití \TeX{}u a \OpTeX/u.
Domnívám se ale, že metodou analogie je schopen i \TeX{}ový nováček vytvořit obvyklý
dokument. Doporučuji mu prostudovat stručný text~\cite[prvni] a uživatelskou 
dokumentaci \OpTeX/u~\cite[optex]. Dokumantace je v \TeX/live k dispozici po
příkazu "texdoc optex".
Na jednotlivé odstavce dokumentace k~\OpTeX/u budu v~této kapitole odkazovat.
Nově je též k dispozici text o základech plain\TeX{}u~\cite[tpp].
Pokročilejší uživatelé \TeX/u mohou použít knihy~\cite[tbn,tst,texbook].


\sec Kapitoly, sekce, podsekce, přílohy

Dokument lze členit na kapitoly, sekce, podsekce a přílohy.
Používají se k~tomu příkazy vysvětlené v~dokumentaci k~\OpTeX/u v~sekci~1.4.1,
Kromě těchto příkazů šablona \ctustyle{} přidává příkaz "\app".
Takže seznam příkazů pro vymezení základní struktury dokumentu vypadá takto:

\begtt
\chap Nadpis kapitoly <ukončený koncem řádku>
\sec Nadpis sekce <ukončený koncem řádku>
\secc Nadpis podsekce <ukončený koncem řádku>
\app Nadpis přílohy <ukončený koncem řádku>
\endtt

Přílohy uvozené příkazem "\app" se chovají stejně jako kapitoly, jen nejsou
číslovány, ale jsou k~nim automaticky vzestupně přiřazena písmena A, B,
C\dots{} Také přílohy mohou být členěny na sekce a podsekce.


\sec Povinné části závěrečné práce

\secc Rozvržení dokumentu

Je doporučeno číslované kapitoly zahájit úvodem, pak další kapitoly podle
potřeby práce a poslední kapitola by měla být označena jako závěr. 
Následně musí být zařazen seznam literatury
(nečíslovaná kapitola) a pak povinná příloha A se zadáním práce 
(pokud není zadaní práce povinně zařazeno před titulní
list nebo těsně za titulní list, viz sekci~\ref[ozadani]).
Poté následují případné další přílohy. Mezi těmito
dalšími přílohami se velmi doporučuje zařadit seznam zkratek a symbolů, jako
zde v příloze~\ref[zkratky].


\secc Literatura

Každá studentská závěrečná práce musí obsahovat seznam použitých zdrojů. 
Toto je jediná nečíslovaná kapitola umístěná na konci textu práce ale před
přílohami. \ctustyle{} nabízí příkaz "\bibchap",
který je potřeba použít místo příkazu "\chap". Příkaz vytvoří záznam pro
obsah, vytiskne slovo Literatura (nebo References v anglicky psané práci).
Toto slovo se generuje automaticky, příkaz zapisujte bez parametru.

Způsob tvorby seznamu literatury a způsob citace na literaturu v~textu je
podrobněji vyložen v~sekci~\ref[citlit].



\sec Obrázky, tabulky, listingy a další

\secc Obrázky

Obrázky ve formátu JPG, PNG (bitmapové) nebo PDF (vektorové i bitmapové) je
možné vložit příkazem "\inspic". Viz sekci 1.6.2 v~dokumentaci k~\OpTeX/u. Pod
obrázek je nutné vložit popisek příkazem "\caption/f", viz sekci~1.4.2 
v~dokumentaci k~\OpTeX/u. \ctustyle{} navíc definuje příkaz "\cinspic", který
umístí obrázek doprostřed. Například:

\begtt
\medskip
\picw=5cm \cinspic cmelak1.jpg
\caption/f Ukázka vložení obrázku na střed, což je asi nejobvyklejší.
\medskip
\endtt
%
vytvoří:

\medskip  \clabel[logo]{Ukázka vložení obrázku na střed}
\picw=5cm \cinspic cmelak1.jpg
\caption/f Ukázka vložení obrázku na střed, což je asi nejobvyklejší.
\medskip

Makro "\cinspic" vyžaduje jméno souboru s příponou ukončené mezerou podobně
jako příkaz "\input". 

Pokud se obrázek vyskytuje dole na stránce tak, že stranu přeplní, nastávají
problémy se stránkovým zlomem. Proto je vhodné obrázky vložit i s~popiskem
do dvojice příkazů "\midinsert" a "\endinsert". V~takovém případě obrázek
implicitně zůstane, kde je, ale při potížích odpluje na začátek následující
stránky:

\begtt
\midinsert
\picw=5cm \cinspic cmelak1.jpg
\caption/f Ukázka vložení obrázku na střed, což je asi nejobvyklejší.
\endinsert
\endtt

Místo příkazu "\midinsert" můžete použít "\topinsert". V~takovém případě
obrázek odpluje na horní část stránky vždy. Raději má vršek aktuální
stránky, ale když to není možné, odpluje na stránku následující.

Chcete-li dostat zmínku o obrázku do seznamu obrázků generovaném na začátku
dokumentu, vložte za "\midinsert" (nebo za "\topinsert") 
"\clabel[<lejblík>]{<popisek>}". Podrobněji je tato vlastnost popsána v
sekci~\ref[aucislo].


\secc Tabulky

Pro tabulky použijte příkaz "\ctable{<deklarace>}{<data>}", 
který je odvozen z~příkazu "\table"
dokumentovaného v~\OpTeX/u v~sekci~1.4.6. \ctustyle{} definuje "\ctable"
tak, že tabulku navíc podkládá modrým pozadím (což je součástí typografického
návrhu šablony) a centruje ji.
Například:

\begtt
...vytvoří tabulku~\ref[absolventiFEL].

\midinsert \clabel[absolventiFEL]{Počet absolventů FEL ČVUT}
\ctable{lrrrrr}{
 \hfil number of       & 2007 & 2008 & 2009 & 2010 & 2011 \crl \tskip4pt
 students Bc. and Mgr. & 6313 & 5913 & 5951 & 5188 & 4737 \cr
 graduate Bc. and Mgr. & 1195 & 1489 & 1379 & 1160 & 1260 \cr
 students Ph.D.        &  457 &  468 &  366 &  395 &  434 \cr
 graduate Ph.D.        &   65 &   60 &   55 &   54 &   51 \cr
}
\caption/t Počet absolventů FEL ČVUT. Tabulka je převzata z~\cite[zyka].
\endinsert
\endtt
%
vytvoří tabulku~\ref[absolventiFEL].

\midinsert \clabel[absolventiFEL]{Počet absolventů FEL}
  \ctable{lrrrrr}{
    \hfil number of       & 2007 & 2008 & 2009 & 2010 & 2011 \crl \tskip4pt
    students Bc. and Mgr. & 6313 & 5913 & 5951 & 5188 & 4737 \cr
    graduate Bc. and Mgr. & 1195 & 1489 & 1379 & 1160 & 1260 \cr
    students Ph.D.        &  457 &  468 &  366 &  395 &  434 \cr
    graduate Ph.D.        &   65 &   60 &   55 &   54 &   51 \cr
}
\caption/t Počet absolventů FEL ČVUT. Tabulka je převzata z~\cite[zyka].
\endinsert

Doporučuji nerámovat tabulky do dalších rámečků, ale využít toho, že tabulka
je automaticky ohraničena modrým podkladem. Je vhodné pouze vložit linku mezi
záhlaví a údaje v~tabulce (viz příkaz "\crl" v~ukázce).

Tabulky (podobně jako obrázky) je vhodné zapouzdřit do dvojice 
příkazů "\midinset" a "\endinsert" nebo "\topinsert" a "\endinsert".


\secc[aucislo] Automaticky číslované objekty

Jak je možné si všimnout, \ctustyle{} automaticky čísluje kapitoly, sekce,
podsekce, dodatky, tabulky, obrázky a pokud uživatel použije "\eqmark", 
očísluje i rovnice. O~tomto číslování a o~odkazech na tato čísla
v~textu pojednávají odstavce 1.4.1 a 1.4.2 v~dokumentaci k~\OpTeX/u. 
Zde jen stručně uvádím,
že číslované objekty je potřeba označit interním lejblíkem příkazem "\label"
a pak je možné na ně odkazovat příkazem "\ref". Existuje ještě možnost
odkazovat na stránku příkazem "\pgref" a na literaturu příkazem "\cite".

Kapitoly se číslují od jedné v~celém dokumentu, sekce se číslují druhým
číslem v~pořadí od jedné v~každé kapitole a podsekce se číslují
třetím číslem od jedné v~každé sekci. Hlubší zanoření (podpodsekce) není
podporováno a není pro studentské práce doporučeno.

Tabulky se číslují od jedné v~každé kapitole a obrázky (nezávisle na
tabulkách) taky. Rovněž rovnice se číslují od jedné v~každé kapitole.
\ctustyle{} volí kompromis mezi krátkým číslováním (Tabulka 27) a dlouhým
číslováním (Tabulka 2.4.6). První extrém nedává představu o kapitole, ve
které je tabulka umístěna, a druhý extrém se čtenáři obtížně pamatuje.

\ctustyle{} definuje kromě příkazu "\label" ještě příkaz
"\clabel[<lejblík>]{<text>}", který funguje jako "\label[<lejblík>]", ale
navíc vloží takto označenou tabulku nebo obrázek do seznamu tabulek nebo
obrázků. Tyto seznamy se vygenerují hned za obsahem dokumentu. Pozor:
není-li tabulka nebo obrázek označen pomocí "\clabel", v~příslušném seznamu
se neobjeví. Někoho může napadnout otázka, proč má psát "<text>" dvakrát:
jednou pro seznam obrázků či tabulek v~příkazu "\clabel" a jednou pod
obrázek v~příkazu "\caption". Je to proto, že ty texty se mohou lišit.
Typicky v~obsahu budou stručnější. Ukázka použití "\clabel" je u~výpisu kódu
k~tabulce~\ref[absolventiFEL].

\OpTeX/ umožňuje použít automaticky číslované odstavce, viz konec sekce
1.4.2 v dokumentaci \OpTeX/u. Je připraveno pět
nezávislých čítačů označených A, B, C, D a E, každý z nich začíná 
v každé kapitole číslovat od jedné. Makro "\numberedpar<čítač>{<slovo>}"
zahájí číslovaný odstavec ve tvaru "<slovo> <číslo kapitoly>.<hodnota čítače>".
Následující příklad deklaruje věty a důsledky číslované společnou řadou
čísel a dále nezávisle číslované definice a příklady.

\begtt
\def\veta     {\numberedpar A{Věta}} 
\def\dusledek {\numberedpar A{Důsledek}} 
\def\definice {\numberedpar B{Definice}} 
\def\priklad  {\numberedpar C{Příklad}} 
\endtt

Po této deklaraci můžete psát "\definice Nechť $M$ je neprázdná ..."
a objeví se odstavec zahájený takto:

\def\definice {\numberedpar B{Definice}} 
\definice Nechť $M$ je neprázdná \dots

Další definice v této kapitole bude mít číslo 2.2, další 2.3 atd. K tomu
mohou být přidány věty a důsledky číslované 2.1, 2.2, atd. Konečně i
příklady v této kapitole budou číslovány 2.1, 2.2, atd. Před takto označené
odstavce lze psát "\label[<lejblík>]" a dá se pak na ně odkazovat pomocí
"\ref[<lejblík>]" a "\pgref[<lejblík>]", tedy odkazování je stejné jako u
všech ostatních automaticky číslovaných objektů.

\secc Listingy, výpisy kódů

Pro listingy, tj. výpisy kódu, použijte dvojici příkazů "\begtt" a "\endtt",
jak o~tom píše dokumentace k~\OpTeX/u v~sekci~1.4.7. \ctustyle{} definuje
"\tthook" tak, aby byly listingy podbarveny světle modrou barvou, což je
součást grafického stylu.

Listingy se lámou do více stránek a jsou tištěny strojopisem, aby to
navodilo atmosféru pohledu do textového programátorského editoru, který
rovněž používá písmo s~pevnou šířkou všech znaků. Pravda, atmosféru to
nevytvoří dokonalou, protože textové editory dnes navíc používají prostředky pro
zvýraznění některých slov (klíčových slov programovacího jazyka atd.).
Chcete-li tedy navodit dokonalou atmosféru, uložte
si zobrazení svého textového editoru jako obrázek a do dokumentu vložte
obrázek. Nebo můžete experimentovat s příkazem "\hisyntax" dokumentovaném v
\OpTeX/u na konci sekce~1.4.7.
Ovšem strohé listingy jen pomocí "\begtt" a "\endtt" jsou velmi doporučené,
protože modrý podklad graficky ladí s~celkovým návrhem \ctustyle{}
a výsledná sazba působí dostatečně střídmě.

Pokud chcete přímo v~odstavci uvádět kusy kódů, obalte je do dvojice 
znaků {\tt\dprime...\dprime}. Je možné tedy psát třeba toto:

\begtt
Chcete-li zdůraznit slovo, použijte {\em kurzívu}, do které
přepnete příkazem "\em", tedy "{\em zvýrazněné slovo}". 
\endtt

Tyto kusy kódu budou uvnitř odstavce tištěny strojopisem a nebudou podléhat
řádkovému zlomu. Bohužel dvojice znaků {\tt\dprime...\dprime} je možné použít jen
uvnitř \uv{obyčejného} odstavce, nikdy nefungují uvnitř parametrů jiných
příkazů (obsahy tabulek, atd.). V~takových místech musíte
do strojopisu přepnout explicitně pomocí "{\tt text}" a pohlídat si sazbu
\TeX{}ovsky citlivých znaků. 
Místo backslashe je možné psát "\bslash" a místo
procenta "\pcent".


\secc Poznámky pod čarou

Pro poznámky pod čarou používejte "\fnote{<text>}" jak je popsáno 
v~sekci~1.2.3 dokumentace k~\OpTeX/u. Vytvoří to poznámku\fnote{Jako je tato.}. 
Poznámky pod čarou jsou číslovány v každé kapitole od jedné. Doporučuji 
s~takovými poznámkami šetřit.

Poznámky na okraji "\mnote", o~kterých také
hovoří dokumentace k~\OpTeX/u, nejsou při použití \ctustyle{} doporučeny.


\secc Zvýrazňování textů

Základní text je psán antikvou (písmem Latin Modern odvozeným z~Computer
Modern). Chcete-li zdůraznit slovo, použijte {\em kurzívu}, do které
přepnete příkazem "\em", tedy "{\em zvýrazněné slovo}". 
Je to obvyklý způsob zdůrazňování, který je typograficky vhodný, protože
netrčí z~textu, ale je viditelný při čtení.

Pokud chcete zdůraznit něco, aby to bylo {\bf vidět z~dálky}, použijte
přepínač "\bf", který při použití \ctustyle{} přepíná do tučného fontu bez
serifů (tj. bez patek). Tedy "{\bf takto}". V~tomto fontu jsou řešeny i
nadpisy. V této verzi \ctustyle{} je přepínačem "\bf" zahájen font
Technika-Bold, který se může někomu jevit jako příliš tučný. 
V~takovém případě můžete použít přepínač "\sbf" (\uv{semibold font}), 
což vypadá {\sbf takto}. Podrobněji o tom je
pojednáno v kapitole~\ref[typo].

Vyznačování podtrháním textu nebo prostrkáním nedoporučuji.


\label[uvozovky]
\secc Uvozovky, pomlčky, nezlomitelné mezery

{\sbf České uvozovky} vypadají \uv{takto}, {\sbf anglické} ``takto''. V~závislosti na
jazyce použijte správné uvozovky. Můžete je napsat přímo v~textovém editoru
(v~UTF-8 kódování), nebo \TeX{}ovsky to uděláte "\uv{takto}" pro češtinu a
"``takto''" pro angličtinu.

{\sbf Pomlčky} v~typografii jsou dvě. 
\begitems
* Střední pomlčka: -- 
(používá se bez mezer kolem ve významu \uv{až} nebo s~mezerami 
jako pomlka ve větě). 
* Dlouhá pomlčka: --- 
(používá se v~anglickém textu). 
\enditems

Můžete tyto znaky napsat přímo v~editoru
v~UTF-8 kódování nebo \TeX{}ovsky: "--" (střední pomlčka), "---" (dlouhá
pomlčka). Čtenář vašeho textu vám strhne nemilosrdně body, pokud ve významu
pomlčky použijete spojovník. Vypadá takto: \uv{-} a promění se na něj
singl znak \clqq"-"\crqq{} ve zdrojovém textu.

{\sbf Nezlomitelná mezera} je mezislovní mezera, ve které nedojde k~zalomení
do řádků. V~\TeX{}ových zdrojových textech se typicky tato mezera značí
vlnkou \clqq"~"\crqq. Existuje program 
"vlna"\urlnote{ftp://math.feld.cvut.cz/olsak/vlna/}, 
který dokáže
zaměnit normální mezery za tyto vlnky ve zdrojovém textu za všemi výskyty
neslabičných předložek, kam skutečně patří v~češtině i slovenštině
nezlomitelná mezera. O~to se tedy uživatel při psaní textu nemusí starat,
jen při závěrečných korekturách použije program "vlna" na všechny vstupní
soubory se zdrojovým textem a spustí \TeX{} znovu. Program "vlna" ovšem nedává
vlnky před čísla citací a referencí a na mnoho míst, kam podle zvyklostí
v~sazbě taky patří. To si musí uživatel pohlídat sám.


\secc Odkazy do internetu.

Do internetu by se nemělo odkazovat přímo v textu, ale 
pomocí poznámky pod čarou "\fnote".
Aby se stalo URL klikatelné a bylo vytištěno správně strojopisem, je nutno
je vložit do parametru příkazu "\url", tedy třeba
"\url{http://petr.olsak.net}" vytvoří \url{http://petr.olsak.net}.
Ovšem navíc je potřeba tento text poslat do poznámky pod čarou. \ctustyle{}
definuje zkratku "\urlnote{<URL text>}", která je totožná 
s~"\fnote{\url{<URL text>}}". Takže text z předchozího odstavce byl napsán
takto:

\begtt
Existuje program "vlna"\urlnote{ftp://math.feld.cvut.cz/olsak/vlna/},
který dokáže...
\endtt

\secc Seznamy

Tvorba seznamů s~odrážkami je popsaná v~sekci 1.4.5 v~dokumentaci
k~\OpTeX/u. (příkazy "\begitems" a "\enditems"). Implicitní odrážku v~seznamu 
definuje \ctustyle{}
jako modrý čtvereček. Podívejte se, jak to vypadá, do 
sekce~\ref[uvozovky] do místa, kde se mluví o~pomlčkách.
Pokud chcete použít seznam v~seznamu,
pro vnitřní seznam použijte "\style x", což vytvoří poněkud menší modré
čtverečky. Pro číslované seznamy použijte "\style n".


\secc Slovníček zkratek

Je možné si například do souboru "glosdata.tex" připravit následující obsah

\begtt
\glos {ČVUT}   {České vysoké učení technické v Praze}                          
\glos {FEL}    {Fakulta elektrotechnická ČVUT}
\glos {FIT}    {Fakulta informačních technologií ČVUT}
\glos {UK}     {Univerzita Karlova}
\glos {MFF}    {Matematicko-fyzikální fakulta UK}
\endtt
%
a zařadit jej do dokumentu na jeho začátek (nejlépe před "\makefront")
pomocí "\input glosdata". Tím se ještě nic nestane. Nyní ale můžete někam do
dokumentu napsat třeba

\begtt
\app Slovníček 
\makeglos
\endtt
%
a v uvedeném místě se objeví slovníček sestavený z "glosdata.tex" a
uspořádaný podle abecedy, třebaže glosdata uspořádána dle abecedy nejsou.
Chcete-li vypnout abecední řazení, pište na začátek dokumentu 
"\let\dosorting=\relax".

Objeví-li se zkratka ze slovníčku někde v dokumentu, můžete ji označit pomocí
příkazu "\glref", například "\glref{ČVUT}", a v tomto místě se vytvoří
hypertextový odkaz do slovníčku.

Je též možné místo "\glref" použít jiné makro "\glosref{<zkratka>}{<význam>}". 
To využijete tehdy, pokud {\em nechcete} mít
souhrnný soubor "glosdata.tex", ale chcete významy jednotlivých zkratek
zapsat až v místě jejich výskytu. To má ale jistá omezení. 
Zatímco zkratka označená "\glref" se v dokumentu může
vyskytovat vícekrát, její význam musí být deklarován pomocí "\glosref" právě jednou
a deklarace významu všech zkratek musejí předcházet místu, kde je slovníček vytištěn
pomocí "\makeglos".

Chcete-li slovníček sestávající z více podsekcí, je nutné jej sestavit
manuálně, jako například v příloze~\ref[zkratky] tohoto dokumentu.

\secc Rejstřík

Rejstřík se u~studentských závěrečných prací nevyžaduje. Nic ale nebrání jej
vytvořit a postupovat přitom podle odstavce~1.5.2 v~dokumentaci k~\OpTeX/u.


\label[citlit]
\sec Citace na literaturu

Odkazy v textu vytváříme příkazem "\cite[<lejblík>]". Lejblíků může být v
hranaté závorce více a jsou odděleny čárkou.
Podrobněji je tato problematika popsána v~dokumentaci k \OpTeX/u
v~sekci~1.5.3.

Seznam použité literatury má být řazen podle pořadí odkazů na citace v~textu.
Toto je součástí zadání, viz přílohu~\ref[zadani]. Osobně se mi to jeví jako nerozumné
rozhodnutí, ale zadání je třeba ctít. 

Jednoltivé položky do seznamu iteratury je možné psát manuálně pomocí
příkazů "\bib". To ale asi není nejlepší nápad.
Pohodlnější možností k vytvoření seznamu literatury 
je přímé čtení databázového souboru ".bib" makry \TeX{}u. 
Stačí si připravit ".bib"
soubor s odpovídajícími údaji (například "mybase.bib")
a do místa, kam má být vložen seznam literatury, napsat

\begtt
\bibchap
\usebib/c (simple) mybase
\endtt

Můžete vyjít z existujícího souboru "mybase.bib" a přidat si tam další položky. Je
tedy možné postupovat metodou analogie. Místo stylu "simple" je možné použít
též styl "iso690".

V seznamu literatury se objeví ze čteného ".bib" souboru jen ty záznamy, 
které jsou v dokumentu
citovány pomocí "\cite". Chcete-li tam přidat další záznamy, je třeba je
\uv{virtuálně citovat} pomocí "\nocite".

K vytvoření ".bib" souboru je vhodné použít například nějaké webové
rozhraní\urlnote{http://www.citace.com}. Doporučuji připravit ".bib" soubor
v kódování UTF-8 a {\em velkým obloukem} se v něm vyhnout
ancientním \TeX{}ovým sekvencím typu "Ol\v{s}\'ak".


\label[ozadani]
\sec Jak vložit zadání práce

Fakulty mají své předpisy, jak a kam vložit zadání bakalářské nebo diplomové
práce. Předpokládejme, že máte zadání v samostatném PDF dokumentu, např. v souboru
"zadani.pdf".

Pokud máte mít zadání práce vloženo jako první nečíslovaný list práce ještě
před titulní stránkou, můžete před "\makefront" vložit:

\begtt
{\nopagenumbers
 \vbox to0pt{\vskip-25mm\centerline{\inspic zadani.pdf }\vss}
 \nextoddpage}
\endtt

Pokud máte mít zadání práce jako druhý list hned za titulním listem,
vložte před "\makefront" následující deklaraci:

\begtt
\specification {%
   \vbox to0pt{\vskip-25mm\centerline{\inspic zadani.pdf }\vss}
}
\endtt

Pokud můžete zadání práce vložit až do přílohy, je možné použít třeba

\begtt
\app Zadání práce
\vbox to0pt{\vskip-25mm\centerline{\inspic zadani.pdf }\vss}
\nextoddpage

\app Další příloha
\endtt


Máte-li dvě varianty zadání (např. ve dvou jazycích) a chcete je dát těsně
za titulní list, můžete psát:

\begtt
\specification {
   \vbox to0pt{\vskip-25mm\centerline{\inspic specifi.pdf }\vss}
   \vfil\break
   \vbox to0pt{\vskip-25mm\centerline{\inspic zadani.pdf }\vss}
}
\endtt

V tomto příkladě se "specifi.pdf" zobrazí na straně třetí a "zadani.pdf" na
straně čtvrté. A stejně jako prve od strany páté pokračuje
poděkování/prohlášení atd.


\sec Pracovní verze dokumentu

Příkazem "\draft" vloženým před příkaz "\makefront" vznikne verze dokumentu
označená datem vzniku a slovem Draft na každé stránce. Je to tedy pracovní
(nefinální) verze.

V~pracovní verzi jsou dále červeně vypsány lejblíky, které jste do dokumentu
vložili pomocí "\label" nebo "\clabel". Jsou umístěny v~místě cíle odkazů.
Při přechodu do finální verze (odstraněním příkazu "\draft") samozřejmě
lejblíky zmizí.

Při tvorbě dokumentu lze využít příkazy "\rfc{<poznámka>}", 
které v sazbě neudělají nic. Je-li ale zapnutý
"\draft", pak se souhrnný seznam těchto "<poznámek>" vypíše na úplně poslední stranu
dokumentu a je zpětně prolinkován s místy, kde byly jednotlivé příkazy "\rfc"
použity. RFC je zkratka za request for correction. Inspirace:

\begtt
\rfc{Tady musím doplnit obrázek}
...
\rfc{Ověřit, zda hodnoty v tabulce jsou OK}
\endtt
 
Jakmile je aktivován "\draft", můžete příkazem "\linespacing=<násobek>"
určit řádkování větší než implicitní řádkování 1. Například
"\linespacing=1.7". Tím se mezi řádky ve výstupním PDF dokumentu 
objeví mezery, do kterých může korektor v~pracovní verzi dokumentu 
vpisovat své poznámky. Při každé změně "\linespacing" je třeba \TeX{}ovat
aspoň dvakrát, aby se srovnalo stránkování v obsahu.

Upozorňuji, že řádkování rozdílné od implicitního řádkování 1, je pouze pro
účely pracovních verzí. Finální verze dokumentu {\em musí} mít řádkování 1. 
Ignorování této zásady bude považováno za nedodržení oficiálního
stylu pro závěrečné práce na ČVUT. Proto taky \ctustyle{} při odstranění příkazu
"\draft" automaticky deaktivuje nastavení "\linespacing".

Velké mezerování mezi řádky bylo dříve
doporučováno pro psaní studentských závěrečných prací, ale všichni lidé, kteří
něco vědí o~typografii, se snaží toto desítky let staré nařízení (vyplývající
z~technologie mechanických psacích strojů a z~normy, podle které autor
odevzdával své rukopisy pořízené na takovém psacím stroji tiskárně) jednoznačně vypudit
jako něco, co nemá při dnešních možnostech pořizování dokumentů žádné
opodstatnění. Typografie je nástroj, kterým předáváme své myšlenky dalším
čtenářům a ten nástroj nesmí čtenáře rušit a unavovat ve čtení.
%
Zmíněná starodávná norma měla za úkol usnadnit tiskárenskému závodu
spočítat počet znaků knihy, které autor dodal v~rukopise, a na základě toho
určit cenu prací. Pokud je potřeba zjistit počet znaků v~současném dokumentu, 
můžete to udělat jednodušeji, například příkazem:

\begtt
pdftotext dokument.pdf - | wc -m
\endtt

Tento příkaz spočítá i znaky v~automaticky generovaném obsahu. Pokud toto
není žádoucí, je možné přepínačem "-f" programu "pdftotext" specifikovat, od
které stránky PDF dokumentu má začít číst. Je třeba tam uvést absolutní číslo
strany PDF dokumentu, nikoli čísla podle stránkových číslic.

Příkaz "\savetoner" umožní vypnout (provizorně při "\draft") modré podklady
pod listingy a tabulkami. Ve finálním dokumentu (po vypnutí "\draft") 
jsou podklady vždy podbarveny.

\sec Finální verze dokumentu

Implicitně se předpokládá tisk na duplexové tiskárně.
Na pravé (liché) straně rozevřené \uv{knihy} zahajuje vždy kapitola 1 a
příloha A. Vlevo od nich pak může být kompletně prázdná strana (tzv. vakát).
Chcete-li zahájit každou kapitolu a každou přílohu na pravé straně
(za cenu případného vakátu vlevo), použijte v záhalaví příkaz "\oddchapters"
(to ale spíše nedoporučuji). Pokud chcete zahájit jen nějakou přílohu vpravo, použijte
"\nextoddpage \app Moje další příloha".

Pokud nechcete použít duplexní tisk a chcete mít každou stranu na
zvláštním papíře, použijte příkaz "\onesideprinting", který přepne záhlaví do
formy vhodné pro jednostranný tisk.

Příkaz "\blackwhite" přepne modrou barvu na šedou. Chcete-li tisknout
nakonec černobíle, je možná lepší použít tuto variantu dokumentu. Pro
finálně vygenerované PDF (které není určeno k tisku) ovšem doporučuji 
vrátit se k barvě.


\sec Prezentace pro zpětný projektor

Chcete-li vytvořit prezentaci ve stejném stylu, podívejte se do souboru
"slides.pdf" a "slides.tex". Soubory slouží jednak jako ukázka,
jak taková prezentace může vypadat, a také poskytují návod, jak takovou
prezentaci vytvořit.

\label[typo]
\chap Poznámky k~typografii

Šablona \ctustyle{} řeší následující věci, které jsou na sobě víceméně 
nezávislé:

\begitems
* Strukturu dokumentu a vymezení jeho povinných částí.
* Způsob, jak vyznačovat jednotlivé části ve zdrojovém textu dokumentu.
* Vzhled výstupu, neboli typografii. To je obsahem této sekce.
\enditems

Vyšel jsem ze zadání v~dodatku~\ref[zadani]. Dále jsem čerpal  
z~{\em Grafických manuálů identity ČVUT}~(původně platný \cite[grafman] a
nově platný~\cite[grafman2]),
které určují nebo určovaly, jak mají vypadat tiskoviny naší univerzity.
V~manuálu je doporučeno střídat kromě černé barvu 
\hbox{\locc\Blue Pantone 300~C (blankytně modrou)} jako výrazný znak tiskovin ČVUT.
Tomuto doporučení jsem vyhověl. Mým cílem bylo oživit typografii závěrečných
prací tak, aby se to s~radostí četlo i psalo. Je v~tom skryto trochu
hravosti a rozpustilosti, ale domnívám se, že jen v~takové míře, v~jaké není
narušen slavnostní ráz a důležitost studentské závěrečné práce.
Navržená šablona může být použita na všech fakultách ČVUT.

Domnívám se, že modrá barva
na monitoru působí dobře a při tisku na černobílé tiskárně se holt
promění v~šedou, ale to je pro prezentaci práce v~tištěné podobě dostačující.
Barevný tisk je navíc pro studenty stále dostupnější.

Předpokládám, že se uživatel nebude v~barvách omezovat, když bude vkládat do
dokumentu schémata a obrázky, ovšem jistou střídmost by měl dodržet.

Oranžová barva (doplňková k~modré) je jen 
\urllink[url:http://petr.olsak.net]{navigační}. Naznačuje čtenáři, že
je vyznačená oblast textu klikací. Tyto oranžové rámečky zcela zmizí při
tisku, protože vytištěný text už pochopitelně klikací není.

Předpokládám, že dokument bude tištěn na duplexové tiskárně, po svázání
listů budou liché stránky vpravo a sudé stránky vlevo.
Podle typografických pravidel mají být vnitřní okraje menší než vnější, ale
šablona je záměrně nastavuje na stejný rozměr, protože vnitřní okraje
se \uv{utopí} ve vazbě závěrečné práce, takže budou nakonec menší (ověřeno).
Toto nastavení má i další výhodu: pro elektronickou verzi dokumentu je lepší
mít stejné okraje. Konečně plovoucí záhlaví je navrženo tak, že svými světle
modrými čtverečky z~vazby jakoby vychází na obě strany k~vnějšímu okraji.

Nevyhověl jsem důsledně doporučení manuálů~\cite[grafman] ani
\cite[grafman2] v~případě fontů, protože toto doporučení bylo v~rozporu
s~požadavkem z~dodatku~\ref[zadani]. Tam se požaduje písmo Latin Modern,
zatímco v~Grafickém manuálu se požadovalo dříve písmo Times a nově písmo
Technika. Nicméně tato verze 3 šablony obsahuje použití písma TechnikaBold 
v~nadpisech, kde působí obstojně. Takže jsem manuálu \cite[grafman2] vyhověl
částečně. Šablona kombinuje Latin Modern jako základní text a Techniku pro
nadpisy a tučné zvýraznění. Je to poněkud odvážná kombinace, jejíž záměrem
bylo zdůraznění extrémního kontrastu: jemné písmo Latin Modern s vlasovými
tahy kombinované s bold variantou Techniky, což dává docela zajímavý
výsledek.

Plné použití {\bf písma Technika} ve všech řezech včetně chlebového písma nebylo
možné, protože Regular varianta písma je pro rozsáhlejší texty nepoužitelná
zejména kvůli nevhodnému duktu. Také platí, že obecně bezserifové písmo není
pro použití v rozsáhlých textech, kde se navíc dají očekávat i matematické
vzorce, vhodné.

Písmo Technika je součástí verze 2 a verze 3 balíčku \ctustyle{}, i když licence písma
není úplně jasná, třebaže jsem se na senátu ČVUT snažil ostatní opakovaně
přesvědčit, že jediná správná licence písma k použití ve studentských
závěrečných pracích je svobodná licence. Takže jsem se kopií písma do této
šablony možná dopustil nějakého prohřešku. Uvidíme. Ve variantě {\sbf
Regular} je toto písmo použito v šabloně ve významu \uv{semibold}, dále je
použito ve variantě {\bf Bold}. Konečně je toto písmo použito ve variantě
{\ssr Book} jako \uv{sans serif}. Ve všech těchto variantách je písmo
zmenšeno na 92~\%, aby se střední výška písma jakž takž shodovala se střední
výškou Latin Modern.

Další změnou ve verzi 2 a 3 této šablony je nové logo ČVUT podle
\cite[grafman2]. Bohužel, lev se nám tím postupně stává ledním medvědem
navíc uvězněným v modré čtvercové kleci. I~proti těmto zvláštnostem
v návrhu loga jsem měl v senátu ČVUT důrazné připomínky%
\urlnote{http://petr.olsak.net/logo-cvut.html},
ale ani to nebylo nic platné.
