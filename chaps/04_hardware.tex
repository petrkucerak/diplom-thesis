
\def\ctustyle{{\ssr CTUstyle}}
\def\ttb{\tt\char`\\} % pro tisk kontrolních sekvencí v tabulkách


\chap Hardware

In this chapter, I would like to specify the hardware requirements, design, and describe the selected microcontroller development kit.

\sec Hardware requirement


The proposed device must support communication through the following interfaces:\fnote{By requirements from the Chap \ref[SystemSpecification].}

\begitems
* An analog interface for reading LED module values,
* Serial communication with the {\it Simatic PLC} (though out of scope for this work, the necessary pins must be included),
* Two Ethernet ports supporting 100Base-TX, functioning as a switch to enable daisy-chain wiring.
* Additionally, the device should manage indication LEDs as a user interface and support the connection of non-SIL peripherals for analysis.
\enditems

The device shall also comply with the SIL2 standard.

\sec Microcontroller options

Based on these options, I looked for different microcontrollers on which to build the device. I shortlisted 5 devices that could be used. I compare four of them in Table \ref[MicrocontrollerComparison].

\midinsert \clabel[MicrocontrollerComparison]{Comparison of suitable Microcontrollers}
\ctable{lll}{
{\sbf Feature} & {\sbf TI Sitara AM335x} & {\sbf STM32MP1} \crl
Analog Inputs & External ADC (SPI/I2C) & Integrated ADC  \cr
\cr
Serial Communication & UART (RS232/RS485) & UART (RS232/RS485) \cr
\cr
Ethernet Ports & 2x 100Base-Tx (Switch) & 2x 100Base-Tx (Switch)  \cr
\cr
Indicator LEDs & GPIO Support & GPIO Support  \cr
\cr
Non-SIL Peripherals & SPI, I2C, USB & SPI, I2C, USB, SDIO  \cr
\cr \cr
{\sbf Feature} & {\sbf NXP i.MX RT1060} & {\sbf Microchip SAM E70} \crl
Analog Inputs & Integrated ADC & Integrated 12-bit ADC \cr
\cr
Serial Communication & UART (RS232/RS485) & UART (RS232/RS485) \cr
\cr
Ethernet Ports & 2x 100Base-Tx (Switch) & 1x 100Base-Tx (Ext.Switch)  \cr
\cr
Indicator LEDs & GPIO Support & GPIO Support  \cr
\cr
Non-SIL Peripherals & SPI, I2C, USB, CAN & SPI, I2C, USB, CAN \cr
}
\caption/t Comparison of suitable Microcontrollers ~\cite[SPRS717K,DS12505, IMXRT1060IEC,DS60001527E]
\endinsert

Two of them meet the requirement for an internal ADC converter, which in the end could not be used anyway, because ADC converters must be able to measure voltages up to 36 V. So you will need to use external converters that can be communicated with via SPI, I2C or other peripherals. Therefore, I discarded the {\it STM32MP1} and the {\it NXP i.MX RT1060} devices whose great advantage was this functionality.

{\it Microchip SAM E70} microcontroller has only one Ethernet port; it would be necessary to connect an external switch, so I rejected it.

As a result, I was deciding between devices from Texas Instrument's Sitara series, specifically the {\it AM335x} and {\it AM243x}. However, the devices from the {\it AM335x} series are built on the Arm Cortex-A8 core, which is unnecessarily powerful for my purposes and is more suited for handling 3D graphics, graphics accelerators, running higher-end operating systems like Android or Linux, etc.~\cite[ow6UXp80FhDCOefy] That's why in the end I opted for the AM243x processor from Texas Instruments. For development, I will use the {\sbf Evaluation Board LaunchPad™ AM-243}.

\sec Microcontroller description

The {\sbf Evaluation Board LaunchPad AM-243x} (LP-AM243x) is equipped with the AM2434 ALX MCU, which is composed of 4x Arm Cortex-R5 and 1x Arm Cortex-M4 cores. The Evaluation Board has a large number of interfaces including Industrial Ethernet, Fast Serial Interface, CAN transceiver, or 512Mb QSOI flash memory. For debugging, the Evaluation Board is equipped with the On-Board debugger XDS110.~\cite[gCEW8O4yxLcIgH5I]

The processor series is similar in architecture, registers, and most features to the AM64x series. Therefore, most of the documentation for these two processor series is the same.

The device is organized into two domains: MAIN and MCU.

\secc The MAIN domain

The {\sbf MAIN domain} is made up of Arm Cortex-R5, which are 4 on the development kit. Actually, it is 2x two real-time cores because it is a Dual-Core Cortex-R5F subsystem. The architecture is based on the ARMv7-R instruction set and supports two configurations at boot time:

\begitems
* {\sbf Dual core mode}: means two independent free-operating cores (Asymmetric Multi-Processing, no coherence).
* {\sbf Single core mode}: one free-operating core and one non-operating core.
\enditems

The cores also allow operating in lockstep mode. {\sbf Lock-step mode} refers to a fault-tolerant operational configuration in which a redundant instance of the CPU logic (and optionally the Accelerator Coherency Port - ACP) executes in parallel with the primary CPU. Both the primary and redundant logic are driven by identical inputs and share the same cache memory, eliminating the need for duplicate cache RAMs. The redundant logic operates synchronously with the functional CPU, replicating its behavior in real time. However, it does not influence the system's outputs or processor behavior directly. Its primary purpose is to enable the detection of faults by comparing the results of the redundant logic with those of the functional CPU.~\cite[8MtYuqxzmRkwRi75] However, I don't use it in my project, as I use the {\it isolated core} option for safe functions. By TI safety documentation, AM243x doesn't support full Lockstep mode.\fnote{Details are described in the {\it Functional Safety for AM2x and Hercules™ Microcontrollers} on the official TI webpage  \url{https://www.ti.com/product/AM2434}.}\cite[mhMhwmghbfIhOJ4A]

The MAIN domain also integrates several subsystems that serve to operate real-time systems, meet industry standards for communication, memory access or inter-core communication. Specifically, these are:
\begitems
* {\sbf Programmable Real-Time Unit} and {\sbf Industrial Communication Subsystem} (PRU-ICSSG), which in addition to real-time support for cores, provides support for industrial standards such as EtherCAT®, PROFINET™, EtherNet/IP™, PROFIBUS®, Ethernet Powerlink™ and SERCOS®.
* 16-bit DDR memory subsystem ({\sbf DDR16}, also referred to as DDRSS) for SDRAM support.
* {\sbf Region-based Address Translation Module} (RAT), which handles the translation of a 32-bit address into a 36-bit output address.
* {\sbf Data Movement Subsystem} (DMSS) for efficient data movement between software, firmware and hardware in all combinations.
* {\sbf Mailbox} (MAILBOX) system for inter-core communication.
* Other standard subsystems to support microcontroller and microcomputer functions such as: {\sbf Spinlock}, {\sbf ADC}, {\sbf GPIO}, {\sbf I2C}, {\sbf SPI}, {\sbf UART}, {\sbf CPSW3G} (3-port Gigabit Ethernet Switch), {\sbf PCIE}, {\sbf Serializer/Deserializer} (SERDES), {\sbf USBSS} (Universal Serial Bus 3.1 Subsystem), {\sbf GPMC} (General Purpose Memory Controller), {\sbf ELM} (Error Location Module), {\sbf FSS with OSPI} (Flash Subsystem with Octal Serial Peripheral Interface), {\sbf MMCSD} (Multi-Media Card/Secure Digital Interface), {\sbf ECAP} (Enhanced Capture Module), {\sbf EPWM} (Enhanced Pulse-Width Modulation Module), {\sbf EQEP} (Enhanced Quadrature Encoder Pulse Module), {\sbf MCAN} (Controller Area Network) to support CAN, {\sbf FSI-RX and FSI-TX} (Fast Serial Interface Receiver and Transmitter) and Timers.
* {\sbf Internal Diagnostics Modules} that provide monitoring and diagnostic functions required to achieve certain safety compliance levels.
\enditems

\secc The MCU domain


The {\sbf MCU domain} is based on the Arm Cortex-M4F, which is seeded once on the development kit. In its documentation, TI refers to the MCU domain as {\it M4FSS Island}, {\it MCU Island}, {\it MCU Channel}, or {\it MCU Subsystem}. I will stick to these terms.

The processor core can be configured as an isolated safety MCU or general-purpose MCU. The processor uses the ARMv7-M instruction set architecture, supports the Nested Vectored Interrupt Controller (NVIC) with 64 inputs, and has the ability to execute code from internal or external memories, etc.\fnote{More details are described in the device documentation \url{https://www.ti.com/lit/ug/spruim2h/spruim2h.pdf?ts=1745386413600}.}

Similarly to the MAIN domain, the MCU domain consists of multiple subdomains in addition to the core, which support standard microcontroller functions in hardware. These include: {\sbf MCU-GPIO}, {\sbf MCU-I2C}, {\sbf MCU-SPI}, {\sbf MCU-UART} and {\sbf MCU Timers}. Like the MAIN domain, MCU island supports {\sbf MCU Internal Diagnostics Modules} which provide monitoring and diagnostic functions required to achieve certain safety compliance levels. The mechanisms are:
\begitems
* One instance of the Dual Clock Comparator (MCU-DCC) module, which is used to determine the accuracy of the clock signal while the application is running.
* One instance of the Error Signaling Module (MCU-ESM) for safety-related events and errors aggregation from throughout the device into one location.
* Multiple ECC aggregator modules supporting ECC mechanisms for providing increased system reliability via reduction of memory software errors by allowing single bit errors to be detected and corrected (SEC) and double bit errors to be detected (DED).
* Memory Cyclic Redundancy Check module used to perform CRC to verify the integrity of a memory system.
\enditems

A limitation of the chip placement on the development board used in this work is that not all pins of the isolated core are accessible. The isolated core exposes only four peripheral interfaces, to which the UART is connected; no other interfaces are available. This presents a minor complication for future development. A more detailed discussion of this issue is provided in the following chapter, in the context of designing the specific system architecture.

The MCU domain is designed to ensure {\sbf isolation from the broader SoC architecture}, incorporating Freedom From Interference (FFI) mechanisms to enhance system reliability and security. The key isolation features include:

\begitems
* Independent interconnect architecture to segregate MCU communication pathways.
* Implementation of firewalls and timeout gaskets to enforce access control and prevent unauthorized interactions.
* Controlled reset isolation to enable independent reset management for the MCU domain.
* Dedicated Phase-Locked Loop (PLL) and Memory-Mapped Register (MMR) control for autonomous clock and configuration management.
* Separate I/O voltage supply rail to ensure electrical isolation and minimize interference from other SoC components.
\enditems

\sec ADC module description

For reading data from the LED modules, I selected an ADC module from Analog Devices, specifically the {\sbf CN0254}. It is a cost-effective, highly integrated 16-bit, 250~kSPS,\fnote{thousands a samples per second} 8-channel data acquisition system that can digitize ±10 V industrial-level signals.~\cite[9nEIxoFYfksIIwGB] The module can communicate using various peripherals such as I2C, SPI, or USB, provided an external converter from Analog Devices is utilized. In my case, I opted to use SPI.

TODO, add schematic of connection

\sec Hardware design

The primary component of the entire system is the aforementioned Sitara AM243x development board. Connected via SPI, an external Analog-to-Digital Converter (ADC) device is responsible for reading values from the LED module. The processor communicates with a Simatic PLC through binary outputs. The device employs LED diodes to display its current status. Additionally, it communicates with a higher-level device, functioning as a gateway, via Ethernet. All system is illustrated in the figure \ref[HWDesign].

\medskip
\clabel[HWDesign]{Hardware Design}
\picw=13cm \cinspic img/04-hw-design.png
\caption/f Hardware design with peripherals
\medskip
