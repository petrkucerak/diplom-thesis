% Lokální makra patří do hlavního souboru, ne sem.
% Tady je mám výjimečně proto, že chci nechat hlavní soubor bez maker,
% která jsou jen pro tento dokument. Uživatelé si pravděpodobně budou
% hlavní soubor kopírovat do svého dokumentu.


\def\ctustyle{{\ssr CTUstyle}}
\def\ttb{\tt\char`\\} % pro tisk kontrolních sekvencí v tabulkách

\chap Introduction

This chapter outlines the motivation behind my thesis and defines its objectives.

\sec Motivation

Rail transport is among the most energy-efficient and high-capacity modes of transportation. Its ability to move large volumes of freight and passengers with relatively low energy consumption makes it a crucial part of global infrastructure. Approximately 8\% of goods and people worldwide are transported by rail.~\cite[M2s1A9T1R76bao5g]  

Given these advantages, ensuring safety and minimizing potential risks are paramount. Before discussing the specific mechanisms that contribute to railway safety, it is essential to justify its significance.

\secc High impact

The low rolling resistance of steel wheels on steel rails, combined with the efficiency of electric propulsion, makes rail transport ideal for moving large volumes of passengers and freight. As shown in Figure~\ref[PassangerCapacity], rail systems provide high passenger capacity relative to time and available space in urban environments.

Rail transport is also highly energy-efficient. Train travel consumes approximately seven times less energy per passenger than car travel in urban settings.~\cite[UZ7ymPeJEhdPijFh] However, due to the scale of rail transport, accidents can result in severe consequences.

\secc Strategic importance

Railways play a critical role in industry by transporting essential supplies such as coal and chemical components, thanks to their high capacity and ability to handle heavy loads. They are also indispensable for military logistics, including the transport of pontoon bridges, tanks, and other infrastructure. In the ongoing conflict in Ukraine, railways remain vital for delivering military aid.~\cite[Gavenda2022] Additionally, rail transport is often a key component in distributing humanitarian aid during natural disasters.

Thus, railway infrastructure must be designed for reliability under various conditions. Safety measures must address both human error and environmental factors. The specific requirements vary by geography—railways in northern regions must withstand extreme cold and heavy snowfall, while those in southern areas need to be resilient against high temperatures and coastal rainfall.

\secc Personal motivation

My primary interest lies in embedded systems, where I can directly control hardware, interact with peripherals, and develop low-level software solutions. The development of safe software for fail-safe platforms aligns with these interests, as it involves ensuring reliability, handling physical constraints, and minimizing abstraction layers.

\medskip
\clabel[PassangerCapacity]{Passenger Capacity of different Transport Modes}
\picw=12cm \cinspic img/01-passenger-capacity.png
\caption/f Passenger Capacity of different Transport Modes. Number of passengers per hour on 3.5 m wide lanes in the city.~\cite[tbBVVJc0f9An81Y4]
\medskip

\sec The aim of this thesis

The aim of this work is to design and implement a system for Railway Fail-Safe Platforms. Specifically, a component for a railroad crossing system that controls LED lights. The device must comply with CENELEC~\fnote{European Committee for Electrotechnical Standardization} standards. More detailed system and functional requirements are discussed in Chapter~\ref[SystemRequirements].

To achieve this, I have outlined the following steps:

\begitems
* Analyze fail-safe software standards for railway systems. Identify the differences between SIL2 and SIL4 platforms and examine specific mechanisms that ensure compliance with safety requirements.

* Choose suitable hardware and design a CENELEC-compliant SIL2+ architecture. The design should include a boot sequence and mechanisms for detecting and mitigating random errors and software faults. The proposed device will consist of a safety part responsible for critical computations and a non-safety part responsible for monitoring and data transmission over Ethernet.

* Implement the designed software architecture on the selected hardware.

* Justify the implementation for SIL2 and evaluate potential steps required to achieve higher SIL levels.
\enditems
