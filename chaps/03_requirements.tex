% Lokální makra patří do hlavního souboru, ne sem.
% Tady je mám výjimečně proto, že chci nechat hlavní soubor bez maker,
% která jsou jen pro tento dokument. Uživatelé si pravděpodobně budou
% hlavní soubor kopírovat do svého dokumentu.

\def\ctustyle{{\ssr CTUstyle}}
\def\ttb{\tt\char`\\} % pro tisk kontrolních sekvencí v tabulkách

\label[SystemRequirements]
\chap System requirements

In this chapter, I would like to specify the system, functional requirements, and interfaces that I want to design and implement. Finally, I will discuss the hazards that need to be addressed or questions that need to be answered when designing the system architecture.

\sec System context

The proposed equipment is part of a level crossing safety system that controls traffic lights and signals. Previously, the system relied on a Simatic PLC~\fnote{\url{https://www.siemens.com/cz/cs/products/automation/systems/industrial/plc.html}} with digital inputs and outputs. However, it lacked the ability to analyze system functionality; issues could only be diagnosed manually using a multimeter. This becomes particularly problematic when the equipment is located in inaccessible areas or exposed to harsh conditions, such as freezing temperatures in Norway.

The goal of the new system version is to integrate a component that enables such analysis, which is precisely the purpose of this device.

The system is designed for use in a country that follows CENELEC standards. The device itself is a road signal monitoring module that controls LED modules on road and train signals based on commands from the Simatic PLC. In addition to managing the signals, it provides diagnostic data about both its own operation and the LED modules, helping to detect faulty components. The device can also transmit monitoring data to a higher-level server.

{\it The term system and device will always mean a component of the overall security system. If this is not the case, this will always be explicitly mentioned.}


\sec System specification

My implemented and proposed project serves as a prototype for the final device. Unlike the final version, it does not need to communicate with the PLC; it only needs to read information from LED modules and transmit the data further.  

In the following diagrams and discussions, I will also include the PLC. This is necessary because its presence must be considered when designing the overall architecture, selecting hardware, and progressing through other development phases.  

The device's safety functions must comply with SIL2, while the final system as a whole is designed to meet SIL4. However, since this prototype is purely a monitoring device and does not introduce any hazards, SIL2 is sufficient.

\medskip
\clabel[SystemDesign]{System design}
\picw=12cm \cinspic img/03-system-desing.png
\caption/f The schematic describes a system with all components.
\medskip

\secc Interface specification

The device shell provides three interfaces:

\begitems
* {\sbf analog interfaces} to LED, that support relevant road signal modules,\fnote{Relevant for this project is modules Silux/Yulux 2.40RS and Silux/Yulux 1.40.}
* interface for {\sbf communication with Simatic PLC}~\fnote{\url{https://www.siemens.com/cz/cs/products/automation/systems/industrial/plc.html}}
* and {\sbf two Ethernet ports} supporting 100Base-Tx, that shall act as a switch to support daisy-chain connection of multiple devices.
\enditems

\secc Function requirements

The whole system function is safety-related to SIL4.\fnote{Safety integrity level 4, detailed describes chap \ref[RAMSCycle]}.

The device shall be able to control 8 LED modules placed on road signals, evaluate the correct operation of the LED module by current measurement, and support configurable flashing operation of LED modules and static illumination.

The device shall provide emergency flashing and steady activation of LED modules in case of control Simatic PLC shutdown.

Device shall provide diagnostic data using simple REST API.

\secc Human interfaces

\secc Safety

\secc Diagnostics

\secc Project limitation

\sec Hazard analysis


My component is related only to SIL2.