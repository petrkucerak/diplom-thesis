% The documentation of the usage of CTUstyle -- the template for
% typessetting thesis by plain\TeX at CTU in Prague
% ---------------------------------------------------------------------
% Petr Olsak  Jan. 2013

% You can copy this file to your own file and do some changes.
% Then you can run:  optex your-file

\input ctustyle3  % The template (in version 3, for OpTeX) is included here.

\worktype [M/EN] % Type: B = bachelor, M = master, D = Ph.D., O = other
                 % / the language: CZ = Czech, SK = Slovak, EN = English

\faculty    {F3}  % Type your faculty F1, F2, F3, etc. or MUVS
            % use main language of your document here:
\department {Department of Measurement}
\title      {Design and Implementation of Systems for Railway Fail-Safe Platforms}
\subtitle   {}
            % \subtitle is optional
\author     {Bc. Petr Kučera}
\authorinfo {Computer Engineering}
\date       {November 2024, May 2025}
\supervisors {doc. Ing. Jiří Novák, Ph.D. and Ing. Bc. Martin Votava}  % One or more supervisors
\studyinfo  {Open Informatics}  % Study programme etc.
\workname   {} % Used only if \worktype [O/*] (Other)
            % optional more information about the document:
\workinfo   {}
            % Title / Subtitle in minor language:
\titleCZ    {Návrh a~implementace systémů pro platformy v~železniční infrastruktuře odolné vůči selhání}
\subtitleCZ {}
            % If minor language is other than English
            % use \titleCZ, \subtitleCZ or \titleSK, \subtitleSK instead it.
\pagetwo    {}  % The text printed on the page 2 at the bottom.

\abstractEN {
   The diploma thesis focuses on the design of a~monitoring system for level crossing signaling using a~multi-core embedded device. The first part analyzes the requirements for systems with functional safety (fail-safe design) in compliance with applicable railway application standards. This is followed by the specification of system requirements, selection of an appropriate hardware platform, and design of a software architecture with an emphasis on safety mechanisms. The third part concentrates on the implementation of a~functional prototype, including the boot process. The final part discusses testing strategies and design approaches that contribute to achieving higher Safety Integrity Levels (SIL). The outcomes of the thesis are a~proposed system architecture and an implemented software prototype ready for testing on the target embedded device.
}
\abstractCZ {
   Diplomová práce se zabývá návrhem monitorovacího systému pro světelnou signalizaci železničního přejezdu s~využitím vícejádrového embedded zařízení. První část práce se věnuje analýze požadavků na systémy s~funkční bezpečností (fail-safe design) v~souladu s~platnými normami pro železniční aplikace. Následuje specifikace systémových požadavků, výběr vhodné hardwarové platformy a~návrh softwarové architektury s~důrazem na bezpečnostní mechanismy. Třetí část se zaměřuje na implementaci funkčního prototypu, včetně bootovacího procesu. Závěrečná část diskutuje strategie testování a~návrhové přístupy, které přispívají ke zvýšení úrovně bezpečnostní integrity (SIL). Výstupem práce je návrh systémové architektury a~implementace softwarového prototypu připraveného k testování na cílovém embedded zařízení.
}           % If your language is Slovak use \abstractSK instead \abstractCZ

\keywordsEN {%
   Functional Safety, Fail-Safe Design, Safety-Critical Development, Railway Crossing, Safety Integrity Level (SIL), Railway Standards, CENELEC, Sitara AM243x, Embedded System, Multi-Core Architecture, Real-Time Core, Isolated Core, Firmware Deployment, SPI Communication, Safety Manual
}
\keywordsCZ {%
   funkční bezpečnost, návrh odolný vůči sehnáním, bezpečně kritický vývoj, železniční přejezd, úroveň integrity bezpečnosti, železniční standardy, CENELC, Sitara AM243x, vestavěné zařízení, více jádrová architektura, jádro reálného času, izolované jádro, nahrávání firmware, SPI komunikace, bezpečnostní manuál
}
\thanks {           % Use main language here
    I express my gratitude to my thesis advisor, doc.~Ing. Jiří Novák, Ph.D., for taking on this project. I also thank Ing.~Bc. Martin Votava from Siemens Mobility for his guidance and consultations. Additionally, I am grateful to Ing. Tomáš Hering and Ing. Jan Volný from Siemens Mobility for their regular consultations. Finally, I extend my thanks to all those not specifically named who contributed to this work, whether through discussions that guided me in the right direction or by providing direct advice on how to proceed.
}
\declaration {      % Use main language here
   The declaration of independent work and the use of artificial intelligence tools, verified and generated in the KOS system, is included in this thesis on a~separate sheet in accordance with regulations. 
   % !!! TODO: Attention, you have to change this item.
   % \signature % makes dots
}

%%%%% <--   % The place for your own macros is here.

%\draft     % Uncomment this if the version of your document is working only.
%\linespacing=1.7  % uncomment this if you need more spaces between lines
                   % Warning: this works only when \draft is activated!
%\savetoner        % Turns off the lightBlue backround of tables and
                   % verbatims, only for \draft version.
%\blackwhite       % Use this if you need really Black+White thesis.
%\onesideprinting  % Use this if you really don't use duplex printing. 

\specification {%
   \vbox to0pt{\vskip-25mm\centerline{\inspic prilohy/zadani1.pdf }\vss}
   \vfil\break
   \vbox to0pt{\vskip-25mm\centerline{\inspic prilohy/zadani2.pdf }\vss}
   \vfil\break
   \vbox to0pt{\vskip-25mm\centerline{\inspic prilohy/zadani3.pdf }\vss}
   \vfil\break
   \vbox to0pt{\vskip-25mm\centerline{\inspic prilohy/prohlaseni.pdf }\vss}
   \vfil\break
}
\makefront  % Mandatory command. Makes title page, acknowledgment, contents etc.

\input chaps/01_introduction
\input chaps/02_problematic-research
\input chaps/03_requirements
\input chaps/04_hardware
\input chaps/05_device-design-and-architecture
\input chaps/06_implementation
\input chaps/07_testing-and-validation
\input chaps/08_conclusion



\input prilohy

\bye
