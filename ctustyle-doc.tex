% The documentation of the usage of CTUstyle -- the template for
% typessetting thesis by plain\TeX at CTU in Prague
% ---------------------------------------------------------------------
% Petr Olsak  Jan. 2013

% You can copy this file to your own file and do some changes.
% Then you can run:  optex your-file

\input ctustyle3  % The template (in version 3, for OpTeX) is included here.

\worktype [M/EN] % Type: B = bachelor, M = master, D = Ph.D., O = other
                 % / the language: CZ = Czech, SK = Slovak, EN = English

\faculty    {F3}  % Type your faculty F1, F2, F3, etc. or MUVS
            % use main language of your document here:
\department {Department of Measurement}
\title      {Design and Implementation of Systems for Railway Fail-Safe Platforms}
\subtitle   {}
            % \subtitle is optional
\author     {Bc. Petr Kučera}
\authorinfo {Computer Engineering}
\date       {November 2024, May 2025}
\supervisors {doc. Ing. Jiří Novák, Ph.D. and Ing. Bc. Martin Votava}  % One or more supervisors
\studyinfo  {Open Informatics}  % Study programme etc.
\workname   {} % Used only if \worktype [O/*] (Other)
            % optional more information about the document:
\workinfo   {}
            % Title / Subtitle in minor language:
\titleCZ    {Návrh a implementace systémů pro platformy v železniční infrastruktuře odolné vůči selhání}
\subtitleCZ {}
            % If minor language is other than English
            % use \titleCZ, \subtitleCZ or \titleSK, \subtitleSK instead it.
\pagetwo    {}  % The text printed on the page 2 at the bottom.

\abstractEN {
   TODO
}
\abstractCZ {
   TODO
}           % If your language is Slovak use \abstractSK instead \abstractCZ

\keywordsEN {%
   TODO
}
\keywordsCZ {%
   TODO
}
\thanks {           % Use main language here
    TODO
}
\declaration {      % Use main language here
   TODO
   % !!! TODO: Attention, you have to change this item.
   \signature % makes dots
}

%%%%% <--   % The place for your own macros is here.

%\draft     % Uncomment this if the version of your document is working only.
%\linespacing=1.7  % uncomment this if you need more spaces between lines
                   % Warning: this works only when \draft is activated!
%\savetoner        % Turns off the lightBlue backround of tables and
                   % verbatims, only for \draft version.
%\blackwhite       % Use this if you need really Black+White thesis.
%\onesideprinting  % Use this if you really don't use duplex printing. 

\specification {%
   %\vbox to0pt{\vskip-25mm\centerline{\inspic prilohy/zadani.pdf }\vss}
}
\makefront  % Mandatory command. Makes title page, acknowledgment, contents etc.

\input chaps/01_introduction
\input chaps/02_problematic-research
\input chaps/03_requirements
\input chaps/04_hardware
\input chaps/05_device-design-and-architecture
\input chaps/06_implementation
\input chaps/07_testing-and-validation
\input chaps/08_conclusion



\input prilohy

\bye
